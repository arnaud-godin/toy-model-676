%HCV transmission toy model structure
%Arnaud Godin
%McGill University
%March 10, 2018

\documentclass[11pt]{article}

\usepackage[round]{natbib}
\usepackage[applemac]{inputenc}
\usepackage[T1]{fontenc}
\usepackage[cyr]{aeguill}
\usepackage{xspace}
\usepackage[top=2.5cm, bottom=2.5cm, left=2.5cm, right=2.5cm]{geometry} 
\usepackage{setspace}
\usepackage{bookman}
\usepackage{graphicx}
\usepackage{fancyhdr}
\usepackage{array}
\usepackage{makecell}
\usepackage{mathtools}
\usepackage{booktabs}
\usepackage{threeparttable}
\usepackage{float}
\usepackage{caption}
\usepackage{listings}
\usepackage{color}
\usepackage{subcaption}
\usepackage{enumerate}
\usepackage{tikz}

\geometry{letterpaper}
\doublespacing
\pagestyle{plain}
\DeclareGraphicsExtensions{.pdf,.jpeg.}
\graphicspath{ {D:/Desktop/CoursesWinter2018/EPIB676} }
\setcounter{secnumdepth}{-1}
\setcellgapes{3pts}
\captionsetup[table]{skip=10pt}
\lstset{language=Python}

\begin{document}
\title{Model structure for HCV transmission among people who inject drugs}
\author{Arnaud Godin}
\date{\today}
\maketitle

\noindent The toy model is composed of three major parts:
\begin{itemize}
	\item HCV transmission among people who inject drugs (PWID)
	\item Incarceration dynamics in PWID
	\item Injecting behaviours
\end{itemize}
 The force of infection is frequency dependent, since it will change with the prevalence of the virus in the population.

\section{\large{Model of HCV transmission among PWID}}
Graphical structure
Description
\begin{align*}
	& \frac{dS(t)}{dt} = -\lambda(t) S(t) + \alpha_{Tx} Tx(t) + \alpha_{A} A(t) \\
	& \frac{dA(t)}{dt} = \lambda(t) S(t) - \sigma_{A} A(t) - \alpha_{A} A(t) \\
	& \frac{dC(t)}{dt} = \sigma_{A} A(t) - \sigma_{C} C(t) - \gamma C(t) \\
	& \frac{dTx(t)}{dt} = \sigma_{C} C(t) - \sigma_{Tx} Tx(t) +
	\omega F(t) - \alpha_{Tx} Tx(t)\\
	& \frac{dF(t)}{dt} = \sigma_{Tx} Tx(t) - \omega F(t) - \gamma F(t) \\
	& N(t) = S(t) + A(t) + C(t) + Tx(t) + F(t) 
\end{align*}

\section{\large{Model of incarceration dynamics among PWID}}
Graphical structure
Description
\begin{align*}
	& \frac{dP_{1}(t)}{dt} = -\eta_{1} P_{1}(t) \\
	& \frac{dP_{2}(t)}{dt} = \eta_{1} P_{1}(t) - \eta_{2} P_{2}(t) + \eta_{4} P_{4}(t) \\
	& \frac{dP_{3} (t)}{dt} = \eta_{2} P_{2}(t) - \eta_{3} P_{3}(t) \\
	& \frac{dP_{4} (t)}{dt} = \eta_{3} P_{3}(t) - \eta_{4} P_{4}(t)
\end{align*}

\section{\large{Model of injecting behaviours}}
Graphical structure
Description


\section{\large{The force of infection: $\boldsymbol{\lambda}(t)$}}
Define the force of infection (see with Charlotte)

\section{\large{Model parameters}}
Table with the most important model parameters to be informed. Find where they will be informed from and propose a prior distribution.

\section{\large{Model fit}}
Describe the fitting procedure and present the results.
\end{document}
